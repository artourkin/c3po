\documentclass[a4paper,12pt]{article}
%\documentclass[a4paper,10pt]{scrartcl}

\usepackage[utf8]{inputenc}


\usepackage{geometry}
\usepackage{graphicx}
\geometry{verbose,tmargin=3cm,bmargin=3cm,lmargin=2cm,rmargin=2cm}
\usepackage[hyphens]{url}
\usepackage[colorlinks,
pdfpagelabels,
pdfstartview = FitH,
bookmarksopen = true,
bookmarksnumbered = true,
linkcolor = blue,
plainpages = false,
hypertexnames = false,
citecolor = black] {hyperref}

\usepackage{multirow}

\usepackage{fancyhdr}

\pagestyle{fancy}

\lhead{Conflict Reduction by Rulebased Postprocessing in C3PO}
%\chead{}
\rhead{}

\lfoot{Lukas Rötzer, Peter Schmidt}
\cfoot{}
\rfoot{\thepage}

\title{Conflict Reduction by Rulebased Postprocessing in C3PO\\ \medskip Concept}
\author{Lukas Rötzer, Peter Schmidt}
\date{25.4.2013}

\pdfinfo{%
  /Title    (Conflict Reduction by Rulebased Postprocessing in C3PO)
  /Author   (Lukas Rötzer, Peter Schmidt)
  /Creator  (Lukas Rötzer, Peter Schmidt)
  /Producer ()
  /Subject  ()
  /Keywords (C3PO, digital preservation, preservation planning, profiling, post processing)
}

\begin{document}

\maketitle
\thispagestyle{empty}

\clearpage

\section{Introduction}

The goal is to develop and apply post-processing methods and tools to \emph{FITS}\footnote{\url{https://code.google.com/p/fits/}} characterisation files within c3po to reduce the problem of "conflicting values" and significantly improve data quality. 

\emph{C3PO}\footnote{\url{http://ifs.tuwien.ac.at/imp/c3po}} already has a way to add Post-Processing rules. There is an interface called PostProcessingRule.java in the c3po-core module, which can be implemented and added in Controller.java. The rules are executed during the import of the \emph{FITS} characterisation files and are applied after they are parsed.

The \emph{FITS} adapter uses an Apache Commons Digester to parse the XML files and to trigger specific events based on rules\footnote{\url{http://commons.apache.org/proper/commons-digester/guide/core.html\#doc.Rules}}.

The \emph{getElements} method pushes a \emph{DigesterContext} on to the Digester-Stack and handles the metadata, which is basically the XML-Stream.
The adaptor rules are used to parse the parameters and the data from the XML file in java objects and properties of the digester context.

Afterwards, when the DigesterContext is completely built from the XML file, the postprocessing method is executed, where:
\begin{itemize}
\item the metadata is set
\item the collection of the db is set
\item if desired, date values can be obtained
\item at last, the postprocessing rules are called
\end{itemize}

\section{Problem Description}

\emph{FITS}, by its design, uses some different tools like \emph{exiftool} or \emph{JHove} to analyse the given data. The problem with this approach is, that these tools sometimes return different values for specific files.

For example, when a HTML file is evaluated, \emph{JHove} returns "Hypertext Markup Language" as format string, whereas \emph{Droid} returns "Extensible Hypertext Markup Language". This results in a conflicted state and should be resolved properly to make it easier, or even possible, to build a strategy for the digital preservation of a file set.

\clearpage

\section{Approach}

The idea is to implement a framework for generic, configurable rules. A way to specify them could be one or several XML files, which are read during run-time.
Each rule would, for example, look out for a set of parameters in elements and change or add some values. So, in principle, these are if-then rules to solve conflicting properties.

To ensure the traceability of the rules, and to confirm, that no valuable knowledge is lost, each change in an element should be logged:

\begin{itemize}
\item Which rules has been applied and in which order were they executed
\item Which values were changed or added
\end{itemize}

This information will be stored in the database. An option to revert the changes could also be implemented, but this is most likely out of the scope of this project

Some possibilities, how rules could work, are explained in the following sections.

\subsection{Equalities}

Often different tools, that are used in fits, have just different names for each and the same file format. For example, \emph{Jhove} classifies HTML files as "Hypertext Markup Language", while \emph{Droid} identifies them as "Extensible Hypertext Markup Language".

Here one could implement some easy matching rules, which compare the different values and, if they are found as equal, will be set to the specified format.
	
\subsection{Priorities}

Sometimes one tool can make a more concrete propostion about the format or mimetype of a file than others. For example, a CSV file is recognized by \emph{Jhove} as a simple plain text file, where \emph{Droid} rightly identified it as "Comma Separated Values".

An approach could be, to assign priorities to different formats and/or values, where one with a higher priority will be chosen over one with a lower priority. Nevertheless such priorities should be assigned carefully and have to be tested properly.

\subsection{Explicit Values}

If the element is in conflicted state, it could also possible to look for other specific values, that are known to exist just on a specific file type or format. From this value one could derive the right format or mimetype.

\end{document}
