\documentclass[a4paper,12pt]{article}
%\documentclass[a4paper,10pt]{scrartcl}

\usepackage[utf8]{inputenc}


\usepackage{geometry}
\usepackage{graphicx}
\geometry{verbose,tmargin=3cm,bmargin=3cm,lmargin=2cm,rmargin=2cm}
\usepackage[hyphens]{url}
\usepackage[colorlinks,
pdfpagelabels,
pdfstartview = FitH,
bookmarksopen = true,
bookmarksnumbered = true,
linkcolor = blue,
plainpages = false,
hypertexnames = false,
citecolor = black] {hyperref}

\usepackage{multirow}
\usepackage{fancyhdr}
\usepackage{listings}
\usepackage{tabularx}
\usepackage{float}

\pagestyle{fancy}

\lhead{Conflict Reduction by Rulebased Postprocessing in C3PO}
%\chead{}
\rhead{}

\lfoot{Lukas Rötzer, Peter Schmidt}
\cfoot{}
\rfoot{\thepage}

\title{Conflict Reduction by Rulebased Postprocessing in C3PO\\ \medskip Report}
\author{Lukas Rötzer, Peter Schmidt}
\date{05.06.2013}

\pdfinfo{%
  /Title    (Conflict Reduction by Rulebased Postprocessing in C3PO)
  /Author   (Lukas Rötzer, Peter Schmidt)
  /Creator  (Lukas Rötzer, Peter Schmidt)
  /Producer ()
  /Subject  ()
  /Keywords (C3PO, digital preservation, preservation planning, profiling, post processing)
}

\begin{document}

\maketitle
\thispagestyle{empty}

\clearpage



\section{Introduction}

The goal was to develop and apply post-processing methods and tools to \emph{FITS}\footnote{\url{https://code.google.com/p/fits/}} characterisation files within \emph{C3PO}\footnote{\url{http://ifs.tuwien.ac.at/imp/c3po}} to reduce the problem of "conflicting values" and significantly improve data quality. 

The problem of conflicting values arises from the fact, that \emph{FITS} uses several independent tools to gather information about the inspected files. This is necessary, because different tools retrieve information of different quality on the vast amount of file types and formats. The downside of this is the fact, that the tools sometimes don't agree on the value of metadata properties. Reasons for this disagreement are various and range from different wording (``Portable Document Format'' vs. ``PDF EXIM'') to different format version recognition, or an insufficient mime-type specification (``text/plain'' vs. ``text/rtf''). 

While some of these conflicts are detected correctly and correspond to corrupted or incorrect data in the inspected file, a bigger portion of them could be avoided, if the specific weaknesses of the used tools were known to \emph{C3PO}. To provide this ability, rules need to be defined by human experts, that know about the abilities and flaws of the tools. They then need to be applied automatically while parsing the data provided by \emph{FITS}. 

\emph{C3PO} already provided a way to refine parsed data with the use of post-processing rules. They are applied either while parsing single properties (pre-processing) or after all information is available (post-processing). We used and adapted these built in mechanisms to apply our conflict resolution framework. The rules can be exchanged and adapted and every user of \emph{C3PO} has the possibility to add his own specific set or remove other rules according to the circumstances of his configuration.

Digital preservation depends strongly on data integrity and authenticity. Traceability of any action taken throughout the whole process is needed not only for testing and quality assurance, but also for provability and justification. To achieve this, we keep track of all rules and changes that were applied to the metadata during conflict resolution. This gives human experts the possibility to evaluate the set of rules used and to redefine them iteratively.

\section{Preliminaries}
When development started, the current stable version of \emph{C3PO} was 0.3 and our concept and implementation was based on that code. In the meantime, \emph{C3PO} 0.4 was released, introducing some redesign especially about parsing and handling command line parameters and configuration. We therefore did not implement configurability of our rule based conflict resolution because that code would have been subject to many changes anyway when porting to the new version.

\section{Workflow}
When profiling a file set, the single files need to be analysed using \emph{FITS}, that generates an XML report for every single file. These reports are parsed by \emph{C3PO} by calling its command line interface with the \emph{-g} option to gather information. Upon initialization, the current configuration is loaded and several threads are spawned to parse the input data in parallel. Every thread uses an adaptor to handle the input data, according to its origin.

To parse the data, \emph{C3PO} implements processing rules, that can be either applied to the data of single XML tags while reading the data (pre-processing) or afterwards, when all data is available in memory.

During pre-processing, 
\begin{itemize}
\item the metadata properties are set
\item the collection name of the database is set
\item if desired, date values can be obtained from the element name
\end{itemize}
Afterwards, when the element is completely built from the XML file, the post-processing methods are executed. At this point, the tree-structured XML representation has been transformed to a flat set of key-value pairs. Finally, the generated object is persisted in the database and the next XML file is fetched form the input.

\section{Rule Framework}

First we considered implementing a special rule engine with its own rule definition language, that would have allowed rather simple rule definitions but would on the other hand be very restrictive and changes to it would result in adoption of the code. Since there exist a great variety of rule frameworks already, we evaluated these and chose to use the \emph{Drools Framework}\footnote{\url{http://www.jboss.org/drools/}} instead, because it's well documented, widely known and has a large community.

Drools is a rule engine, based on the \emph{Rete algorithm} and tailored for the Java language. Rete was adopted to an object-oriented interface, which allows for more natural expressions of business rules.construct

We added a new PostProcessingRule called \emph{DroolsConflictResolutionProcessingRule} to the workflow, which takes a single element for processing and puts it into working memory of the Drools engine. We used Drools in a stateless way, so that from the outside it acts like a black box, that modifies the inserted element directly. Every element is handled separately. Because of the parallel nature of the processing mechanism of \emph{C3PO}, every thread uses its own separated working memory. Drools now investigates the provided element and tries to find rules with matching left-hand-sides (LHS) and executes the imperative right-hand-side code. If a rule marks an element as modified - despite of changing any values - the rules are re-evaluated according to their priority (or ``salience''). When no more matching rules are found, Drools stops processing and the (possibly modified) element is returned to the adaptor.

\subsection{Rules}

With the Drools rule framework one can write a variety of rules, from very simple to large and complicated ones. Detailed instructions can be found in the Drools expert documentation\footnote{\url{http://docs.jboss.org/drools/release/5.5.0.Final/drools-expert-docs/html_single/index.html}}.

\subsubsection{Basic Rules Structure}

A rule consists of of the following parts, but can basically seen as an IF-THEN conditional construct:
\begin{itemize}
\item \emph{name}: identifies a rule
\item \emph{salience}: ranges from 0 to 1000, where 1000 is the maximum priority
\item \emph{when}: a list of (implicitly conjunctive) combined patterns defines, when a rule is triggered
\item \emph{then}: Java-like imperative code defines, what to do, when the rule is triggered
\end{itemize}

LHS patterns allow variable binding, bean-like access to nested properties and some aggregation functions.

A very basic rule looks like the following:
\lstset{xleftmargin=\parindent,basicstyle=\footnotesize\ttfamily,morecomment=[l]{//} }
\begin{lstlisting}

rule "set format GZIP to GZIP Format from Exiftool"
        salience 160
    when 
        $e : Element()
        $md : MetadataRecord(
            property.id == "format", 
            value == "GZIP",
            util.isFromTool(this, "Exiftool")
        ) from $e.metadata

    then
        String newValue = "GZIP Format";
        $md.setValue(newValue);
        modify ($e) {
            getId()
        }
end
\end{lstlisting}

This rule triggers, when
\begin{enumerate}
 \item an element exists (and binds it to $\$e$)
 \item if the element $\$e$ has any MetadataRecord in its $metadata$ property (a list of MetadataRecords), that suffices the following conditions:
    \begin{enumerate}
      \item its properties ID is equal to ``format'' and
      \item its value is equal to ``GZIP'' and
      \item the execution of the external Java method $isFromTool$ on the (global) variable $util$ returns true for the given MetadataRecord and the parameter ``Exiftool''.
    \end{enumerate}
\end{enumerate}

The used global variable is used to provide a utility object to the rules engine that offers some useful methods and reduce the complexity of the rules LHS. This specific method checks if a metadata record was generated by a tool named ``Exiftool'', regardless of other sources or its version.

If an element suffices these conditions, the RHS is executed which sets the records value to ``GZIP Format'' and finally marks the element $\$e$ as modified. 

This example further shows a problem when using Drools in this way: The \emph{modify} block can only be used on variables that are known to the working memory. Because it only contains the current element and has no direct knowledge about its nested properties, the modify block cannot be called on the metadata record directly. But not notifying Drools about a change of the elements state would not trigger a re-evaluation of other rules. Therefore this ``empty'' modify block needs to be used to signalise a change of the element.

\subsubsection{General Strategies}

There exist several different strategies on how a conflict can be resolved. This mainly depends on the type of conflict, e.g. is it just a naming conflict or does one or several tools report a false value.

\begin{itemize}
\item \textbf{Update}: When a tool is known to report an unusual string representation for a fact, this difference causes conflicts (e.g. Jhove or NLNZ report the sampling\_frequency\_unit as ``in.'' while other tools use ``inches'' - although they mean the same fact, it is a conflict for \emph{C3PO}). By simply changing these values, conflicts can be resolved.
\item \textbf{Ignore}: When it's known, that a tool sometimes reports false values on a specific metadata element (especially when other tools report different values), then these false values can just be ignored. (For example, if Exiftool reports some format string, and at least two others claim it to be ``Portable Document Format``, Exiftool can be ignored because its known to have errors detecting PDFs.)
\item \textbf{Merge (and Resolve)}: When several tools agree on a value of a metadata element, but are marked as conflicting (maybe from the initial FITS information or after conflicting values have been changed by prior updates), the values can be merged. Merging records simply means adding all sources of one record to another and removing the first one. If no differing records are left, the conflict has been resolved.

\end{itemize}

\subsubsection{Basic C3PO Rule Framework}
Drools supports separation of rules in multiple files. Therefore we provide basic rules that are necessary for the conflict resolution to work without resolving conflicts on its own in a separated file. Exemplary extensions for conflict resolution rules are provided in two further files. In the future this set of additional files can be partly provided as a standard or default resolution strategy and can be extended by expert users on demand by adding user generated rule files.

The basic rule set consists of five rules:
\begin{enumerate}
 \item \textbf{Update resolved conflicts}: If all metadata records for a given property are marked as conflicting, but all values are equal, merge them to a single record and update the status to ''SINGLE\_VALUE``. This rule has the highest priority of 999, so it is always applied first after an element got changed.
 \item \textbf{Merge conflicting records of equal value}: If not all but some conflicts on some property could be resolved, the records reporting the same value are merged together. This is a convenience rule to ensure that distinct conflicting records have distinct values. It has the second highest priority of 998.
 \item \textbf{Split conflicting record with multiple sources}: If no more matching rules are found, this rule (priority 1) takes any metadata record that is conflicting and has more than one source. This metadata record is then split up into several records, one for each source. This is the inverse operation of the merge rule and is necessary, because the \emph{C3PO} backend datastructure is not able to persist more than one source per record if that record is conflicting. The splitting is not only necessary because of prior merging, but also because the intially parsed FITS files sometimes contain such merged conflicting results (see Analysis below). It is important that this rule is does not mark the element as modified to avoid loops between the merge and the split rule.
 \item \textbf{Report conflicts}: At the very end of rules processing (priority 0), an element that has metadata records is reported for later analysis.
 \item \textbf{Report non conflicting elements}: If the element has no conflicts, this rule (priority 0) is triggered which is mostly used for debugging and analysis.
\end{enumerate}


\subsection{Modification Logging}

To ensure the traceability of the rules, and to confirm, that no valuable knowledge is lost, each change to an element is logged:
\begin{itemize}
\item Which rules has been applied (by name) and in which order were they executed.
\item What was the original value, before it got changed by the rule.
\end{itemize}

To achieve this, we extended the datamodel of an element by adding a list of log-entries and implemented a listener that reacts on changes made to data in the working memory (indicated by the use of the $modify$ block on the RHS of a rule). This ElementModificationListener adds log entries to the modified element, which are finally stored in database with the element.

\subsection{Rule Activation, Conflict Logging and Activity Tracing}

A listener was implemented that monitors activation of rules and increments a counter for every rule on activation. This is helpful for analysing the rules performance and the quality of the parsed FITS files.

Further, a conflict collector is used to collect data about unresolved conflicts that are reported by the final rule mentioned above. It maintains a map of conflicting properties associated with the corresponding elements. This data is essential when creating new rules or updating existing ones because it allows a deeper inspection of remaining conflicts.

Because the elements are processed by several threads in parallel, tracing rule activation and modifications with a simple logger is hardly possible. We therefore provided a log collector that takes log messages from the rules RHS and modification listener. These messages are then forwarded to the applications logging system (currently Sysout) at once to keep relevant data together. The messages are passed to the log collector with a numerical log level that allows filtering the output beforehand.

\section{Analysis}

For deriving and analysing our rules, we used the 10 subsets of govdocs\footnote{\url{http://digitalcorpora.org/corp/nps/files/govdocs1/zipfiles/}}.
Since the distribution of triggered rules regarding the various conflicts is similar, we only consider the subsets zero to four in our analysis.
As can bee seen in \ref{tab:files}, from 4986 files in total, 2438 had conflicts after an initial run of \emph{C3PO}. After implementing and activating our rules, 1244 files remained in conflict, which corresponds to a decrease of nearly 49\%. The number of remaining conflicts could of course been reduced further by adding new rules to the system.

Since one element can have multiple conflicts, we also analysed the impact of our implementation and rule set on the total number of conflicts.
After an initial run of \emph{C3PO}, 5118 conflicting metadata properties were reported, where multiple conflicts on the same metadata property of the same element are counted only once. This number was reduced to 2377, which is a decrease of over 53\%.

A detailed itemization of the different rules and how much conflicts they resolved can be seen in table \ref{tab:conflicts}.

\begin{table}[ht]
\begin{center}

\begin{tabular}[h]{l||r|r|r|r|r||r}
 &  \multicolumn{5}{c}{Datasets} \\
        & 0 & 1 & 2 & 3 & 4 & Total \\
\hline
Total Files & 998 & 998 & 997 & 997 & 996 & 4986\\ 
\hline
with conflicts (pre) & 477 &	492 & 485 & 482	& 502 &	2438\\ 
with conflicts (post) & 249	& 243 &	240 & 253 &	259 & 1244\\
\hline
resolved files & 228 & 249 & 245 & 229 & 243 & 1194 \\
individual conflicts & 533 & 563 & 565 & 515 & 565 & 2741 \\


\end{tabular}
\end{center}
\caption{Analysis of improvements in files}
\label{tab:files}
\end{table}

\begin{table}[H]
\begin{center}
\begin{tabular}[h]{l||r|r|r|r|r||r}
rule activations &  \multicolumn{5}{c}{Datasets} \\
        & 0 & 1 & 2 & 3 & 4 & Total \\
\hline
set 'GZIP' to 'GZIP Format' & 7 & 14 & 13 & 18 & 14 & 66\\
set 'application/rtf' mimetype to 'text/rtf' & 1 & 1 & 0 & 3 & 0 & 5\\
set 'RTF' format to 'Rich Text Format' & 1 & 1 & 0 & 3 & 0 & 5\\
ignore 'text/plain' if more precisise 'text/*' available & 18 & 22 & 20 & 23 & 14 & 97\\
set format 'XLS' to 'Microsoft Excel Format' & 57 & 57 & 54 & 55 & 58 & 281\\
ignore Droid 'PPT' false positives & 0 & 1 & 0 & 0 & 1 & 2\\
ignore others mimetypes if Droid reports PPT & 7 & 2 & 3 & 2 & 3 & 17\\
ignore others formats if Droid reports PPT & 63 & 58 & 48 & 46 & 60 & 275\\
if Jhove and Droid report xhtml, ignore others & 1 & 5 & 6 & 4 & 7 & 23\\
ignore Jhove 'text/html' if 'application/xhtml+xml' & 24 & 22 & 20 & 21 & 22 & 109\\
Resolve Jhove 'HTML Transitional' format string & 40 & 38 & 46 & 32 & 37 & 193\\
html format version (etc) to value without prefix & 92 & 86 & 110 & 78 & 78 & 444\\
set sampling frequency unit by Jhove & 32 & 29 & 34 & 40 & 39 & 174\\
set sampling frequency unit by NLNZ & 72 & 62 & 70 & 68 & 72 & 344\\
ignore fractional exposure time by NLNZ & 14 & 7 & 12 & 10 & 10 & 53\\
\hline
initial conflicts & 1016 & 1010	& 1037 & 994 &1061 & 5118 \\
\hline
remaining conflicts & 483 &	447	& 472 &	479	& 496 &	2377

\end{tabular}
\end{center}
\caption{Analysis of improvements in conflict resolution}
\label{tab:conflicts}
\end{table}

\subsection{Detected Flaws in C3PO/FITS}
During analysis of the used dataset and the impact of our rules on it, we recognized some fundamental flaws, that were mentioned before and shall be explained here.

First, we analysed the dataset by only applying our reporting rules to get an overview of the dataset. Further by only applying our basic rule for updating the status if a reported conflict was no longer existent, the number of conflicts dropped by 15 to 20 \% (17 \% on the datasets 0 to 4). So obviously FITS and/or \emph{C3PO} reported conflicts where no differing values were present. 

This is mainly (but not solely) because FITS treats the format, mimetype and format version as a triple. If one of these properties is conflicting, the others are marked as conflicting as well. After parsing the files provided by FITS, \emph{C3PO} flattens the datastructure and breaks up the triple, but does not update the status. Beside this phenomenon, also plain text properties like 'author' were marked as conflicting, while the provided values were identical.

Further, beside the unresolved initial conflicts, a more severe flaw of \emph{C3PO} was detected during code analysis: The backend datastructure stores a list of values and a list of sources per metadata record. If the record has a single value, only one value is stored and with it a list of sources that reported the value. On the other hand, if a conflict is present, values and sources are stored in two lists and are only ''linked`` by preserving their respective order. This introduces the problem, that if a conflicting record has multiple sources, only the first of them can be stored. While during processing of the elements in \emph{C3PO} all information is available in memory, essential data (additional sources on conflicting records) is lost as soon as the elements are persisted.

This problem was even aggravated by our merging rule described earlier. To counter this problem we added the rule to split up conflicting metadata records if they were based on multiple rules.

The following table shows the impact of the problems stated above:

\begin{table}[ht]
\begin{center}

\begin{tabular}[h]{l||r|r|r|r|r||r}
 &  \multicolumn{5}{c}{Datasets} \\
        & 0 & 1 & 2 & 3 & 4 & Total \\
\hline
Total Files & 998 & 998 & 997 & 997 & 996 & 4986\\ 
\hline
conflicts on equal values & 160 & 196 & 181 & 158 & 190 & 885 \\
conflicting records, multiple sources & 184 & 246 & 248 & 234 & 234 & 1146\\

\end{tabular}
\end{center}
\caption{Impact of flaws in FITS/\emph{C3PO} on the initial dataset }
\label{tab:files}
\end{table}


\end{document}
