\documentclass[a4paper,12pt]{article}
%\documentclass[a4paper,10pt]{scrartcl}

\usepackage[utf8]{inputenc}


\usepackage{geometry}
\usepackage{graphicx}
\geometry{verbose,tmargin=3cm,bmargin=3cm,lmargin=2cm,rmargin=2cm}
\usepackage[hyphens]{url}
\usepackage[colorlinks,
pdfpagelabels,
pdfstartview = FitH,
bookmarksopen = true,
bookmarksnumbered = true,
linkcolor = blue,
plainpages = false,
hypertexnames = false,
citecolor = black] {hyperref}

\usepackage{multirow}

\usepackage{fancyhdr}

\usepackage{listings}
\usepackage{tabularx}

\pagestyle{fancy}

\lhead{Conflict Reduction by Rulebased Postprocessing in C3PO}
%\chead{}
\rhead{}

\lfoot{Lukas Rötzer, Peter Schmidt}
\cfoot{}
\rfoot{\thepage}

\title{Conflict Reduction by Rulebased Postprocessing in C3PO\\ \medskip Report}
\author{Lukas Rötzer, Peter Schmidt}
\date{05.06.2013}

\pdfinfo{%
  /Title    (Conflict Reduction by Rulebased Postprocessing in C3PO)
  /Author   (Lukas Rötzer, Peter Schmidt)
  /Creator  (Lukas Rötzer, Peter Schmidt)
  /Producer ()
  /Subject  ()
  /Keywords (C3PO, digital preservation, preservation planning, profiling, post processing)
}

\begin{document}

\maketitle
\thispagestyle{empty}

\clearpage



\section{Introduction}

The goal was to develop and apply post-processing methods and tools to \emph{FITS}\footnote{\url{https://code.google.com/p/fits/}} characterisation files within \emph{C3PO}\footnote{\url{http://ifs.tuwien.ac.at/imp/c3po}} to reduce the problem of "conflicting values" and significantly improve data quality. 

The problem of conflicting values arises from the fact, that \emph{FITS} uses several independent tools to gather information about the inspected files. This is necessary, because different tools retrieve information of different quality on the vast amount of file types and formats. The downside of this is the fact, that the tools sometimes don't agree on the value of metadata properties. Reasons for this disagreement are various and range from different wording (``Portable Document Format'' vs. ``PDF EXIM'') to different format version recognition, or an insufficient mime-type specification (``text/plain'' vs. ``text/rtf''). 

While some of these conflicts are detected correctly and correspond to corrupted or incorrect data in the inspected file, a bigger portion of them could be avoided, if the specific weaknesses of the used tools were known to \emph{C3PO}. To provide this ability, rules need to be defined by human experts, that know about the abilities and flaws of the tools. They then need to be applied automatically while parsing the data provided by \emph{FITS}. 

\emph{C3PO} already provided a way to refine parsed data with the use of post-processing rules. They are applied either while parsing single properties (pre-processing) or after all information is available (post-processing). We used and adapted these built in mechanisms to apply our conflict resolution framework. The rules can be exchanged and adapted and every user of \emph{C3PO} has the possibility to add his own specific set or remove other rules according to the circumstances of his configuration.

Digital preservation depends strongly on data integrity and authenticity. Traceability of any action taken throughout the whole process is needed not only for testing and quality assurance, but also for provability and justification. To achieve this, we keep track of all rules and changes that were applied to the metadata during conflict resolution. This gives human experts the possibility to evaluate the set of rules used and to redefine them iteratively.

\section{Preliminaries}
When development startet, the current stable version of \emph{C3PO} was 0.3 and our concept and implementation was based on that code. In the meantime, \emph{C3PO} 0.4 was released, introducing some redesign especially about parsing and handling command line parameters and configuration. We therefore did not implement configurability of our rule based conflict resolution because that code would have been subject to many changes anyway when porting to the new version.

\section{Workflow}
When profiling a file set, the single files need to be analysed using \emph{FITS}, that generates an XML report for every single file. These reports are parsed by \emph{C3PO} by calling its command line interface with the \emph{-g} option to gather information. Upon initialization, the current configuration is loaded and several threads are spawned to parse the input data in parallel. Every thread uses an adaptor to handle the input data, according to its origin.

To parse the data, \emph{C3PO} implements processing rules, that can be either applied to the data of single XML tags while reading the data (pre-processing) or afterwards, when all data is available in memory.

During pre-processing, 
\begin{itemize}
\item the metadata properties are set
\item the collection name of the database is set
\item if desired, date values can be obtained from the element name
\end{itemize}
Afterwards, when the element is completely built from the XML file, the post-processing methods are executed. At this point, the tree-structured XML representation has been transformed to a flat set of key-value pairs. Finally, the generated object is persisted in the database and the next XML file is fetched form the input.

\section{Rule Framework}

First we considered implementing a special rule engine with its own rule definition language, that would have allowed rather simple ruledefinitions but would on the other hand be very restrictive and changes to it would result in adaption of the code. Since there exist a great variety of rule frameworks already, we evaluated these and chose to use the \emph{Drools Framework}\footnote{\url{http://www.jboss.org/drools/}} instead, because it's well documented, widely known and has a large community.

Drools is a rule engine, based on the \emph{Rete algorithm} and tailored for the Java language. Rete was adopted to an object-oriented interface, which allows for more natural expressions of business rules.

We added a new PostProcessingRule called \emph{DroolsConflictResolutionProcessingRule} to the workflow, which takes a single element for processing and puts it into working memory of the Drools engine. We used Drools in a stateless way, so that from the outside it acts like a black box, that modifies the inserted element directly. Every element is handled seperately. Because of the parallel nature of the processing mechanism of \emph{C3PO}, every thread uses its own seperated working memory. Drools now investivates the provided element and tries to find rules with matching left-hand-sides (LHS) and executes the imperative right-hand-side code. If a rule marks an element as modified - dispite of changing any values - the rules are reevaluated according to their priority (or ``salience''). When no more matching rules are found, Drools stops processing and the (possibly modified) element is returned to the adaptor.

\subsection{Rules}

With the Drools rule framework one can write a variety of rules, from very simple to large and complicated ones. Detailed instructions can be found in the Drools expert documentation\footnote{\url{http://docs.jboss.org/drools/release/5.5.0.Final/drools-expert-docs/html_single/index.html}}.

\subsubsection{Basic Rules Structure}

A rule consists of of the following parts, but can basically seen as an IF-THEN conditional construct:
\begin{itemize}
\item \emph{name}: identifies a rule
\item \emph{salience}: ranges from 0 to 1000, where 1000 is the maximum priority
\item \emph{when}: a list of (implicitly conjunctive) combined patterns defines, when a rule is triggered
\item \emph{then}: java-like imperative code defines, what to do, when the rule is triggered
\end{itemize}

LHS patterns allow variable binding, access to nested properties and some aggregation functions.

A very basic rule looks like the following:
\lstset{xleftmargin=\parindent,basicstyle=\footnotesize\ttfamily,morecomment=[l]{//} }
\begin{lstlisting}

rule "set format GZIP to GZIP Format from Exiftool"
        salience 160
    when 
        $e : Element()
        $md : MetadataRecord(
            property.id == "format", 
            value == "GZIP",
            util.isFromTool(this, "Exiftool")
        ) from $e.metadata

    then
        String newValue = "GZIP Format";
        $md.setValue(newValue);
        modify ($e) {
            getId()
        }
end
\end{lstlisting}

This rule triggers, when the element has a MetadataRecord ``format`` with a value ``GZIP``, which was derived from ``Exiftool``. If this the case, the value is set ``GZIP Format`` to match the typical value derived from other tools.
The modify function is called, to signalise a change of the element.

\subsubsection{Strategies}

There exist several different strategies on how a conflict can be resovled. This mainly depends on the type of conflict, e.g. is it just a naming conflict or does one or several tools report a false value.

\begin{itemize}
\item \textbf{Merge}: When several tools agree on a value of a metadata element, but just have different string representation, the values can be merged. For example, Exiftool reports ``XLS`` and all other tools report ``Microsoft Excel Format``, then the value is set to ``Microsoft Excel Format``
\item \textbf{Ignore}: When it's known, that a tool sometimes reports false values on a specific metadata element, but other tools typically generate the correct value, then these false values can just be ignored. For example, if Exiftool reports some format string, and at least to others claim it to be ``Portable Document Format``, Exiftool can be ignored.

\end{itemize}


\subsection{Logging}

To ensure the traceability of the rules, and to confirm, that no valuable knowledge is lost, each change to an element is logged:
\begin{itemize}
\item Which rules has been applied and in which order were they executed
\item Which values were changed or removed
\end{itemize}

To achieve this, we extended the datamodel of an element by adding a list of log-entries. We also implemented two listeners called ``ElementModificationListener`` and ``RuleActivationListener``. The RuleActivationListener helps keeping track of activated rules by printing the statistics to the debug output.
The ElementModificationListener instead watches elements and adds modifications to the list of log-entries, which will finally be stored in database, when the element is persisted.


\section{Analysis}

For deriving and analysing our rules, we used the 10 subsets of govdocs\footnote{\url{http://digitalcorpora.org/corp/nps/files/govdocs1/zipfiles/}}.
Since the distribution of triggered rules regarding the various conflicts is similar, we only consider the subsets zero to four in our analysis.
As can bee seen in \ref{tab:files}, from 4986 files in total, 2438 had conflicts after an initial run of C3PO. After implementing and activating our rules, 1244 files remaind in conflict, which corresponds to a decrease of nearly 49\%.

Since one file can have multiple conflicts, we also analysed the impact of our implementation and rule-set on the total number of conflicts.
After an initial run of C3PO we had 5118 conflicts. This number was reduced to 2377, which is a decrease of over 53\%.
A detailed itemization of the different rules and how much conflicts they resolved can be seen in table \ref{tab:conflicts}.

\begin{table}[ht]
\begin{center}

\begin{tabular}[h]{l||r|r|r|r|r||r}
Files &  \multicolumn{5}{c}{Datasets} \\
        & 0 & 1 & 2 & 3 & 4 & Total \\
\hline
Total & 998 & 998 & 997 & 997 & 996 & 4986\\ 
\hline
with conflicts (pre) & 477 &	492 & 485 & 482	& 502 &	2438\\ 
with conflicts (post) & 249	& 243 &	240 & 253 &	259 & 1244\\
\hline
resolved & 228 & 249 & 245 & 229 & 243 & 1194

\end{tabular}
\end{center}
\caption{Analysis of improvements in files}
\label{tab:files}
\end{table}

\begin{table}[ht]
\begin{center}
\begin{tabular}[h]{l||r|r|r|r|r||r}
conflicts / improvements &  \multicolumn{5}{c}{Datasets} \\
        & 0 & 1 & 2 & 3 & 4 & Total \\
\hline
conflicts & 1016 & 1010	& 1037 & 994 &1061 & 5118 \\
\hline
GZIP to GZIP Format & 7 & 14 & 13 & 18 & 14 & 66\\
set application/rtf mimetypes to text/rtf & 1 & 1 & 0 & 3 & 0 & 5\\
set RTF format to Rich Text Format & 1 & 1 & 0 & 3 & 0 & 5\\
ignore text/plain if more precisise text/* available & 18 & 22 & 20 & 23 & 14 & 97\\
set format XLS to Microsoft Excel Format & 57 & 57 & 54 & 55 & 58 & 281\\
ignore Droid PPT false positives & 0 & 1 & 0 & 0 & 1 & 2\\
ignore others mimetypes if Droid reports PPT & 7 & 2 & 3 & 2 & 3 & 17\\
ignore others formats if Droid reports PPT & 63 & 58 & 48 & 46 & 60 & 275\\
if Jhove and Droid report xhtml, ignore others & 1 & 5 & 6 & 4 & 7 & 23\\
ignore Jhove text/html if application/xhtml+xml & 24 & 22 & 20 & 21 & 22 & 109\\
Resolve Jhove HTML Transitional format string & 40 & 38 & 46 & 32 & 37 & 193\\
html format version (etc) to value without prefix & 92 & 86 & 110 & 78 & 78 & 444\\
set sampling frequency unit Jhove & 32 & 29 & 34 & 40 & 39 & 174\\
set sampling frequency unit NLNZ & 72 & 62 & 70 & 68 & 72 & 344\\
ignore fractional exposure time by NLNZ & 14 & 7 & 12 & 10 & 10 & 53\\
\hline
remaining conflicts & 483 &	447	& 472 &	479	& 496 &	2377

\end{tabular}
\end{center}
\caption{Analysis of improvements in conflict resolution}
\label{tab:conflicts}
\end{table}


\end{document}
