\documentclass{beamer}

\usepackage[utf8x]{inputenc}
\usepackage{default}
\usepackage[german]{babel}
\usepackage{pgfpages}
\usepackage{listings}

\usetheme{Frankfurt}
\usecolortheme{default}
\useinnertheme{rounded}
% | default | inmargin |
%	rectangles | rounded
%}

\setbeamercovered{transparent}
\setbeamertemplate{footline}[frame number]
%\setbeameroption{show notes on second screen=right}
%\setbeameroption{hide notes}

\title{Conflict Reduction by Rulebased Postprocessing in C3PO
}

\author[P. Schmidt]{
	Peter Schmidt\\ 
	Lukas Rötzer
}

\begin{document}

\frame{
  \titlepage
}

\section{Introduction}

\begin{frame}{Introduction}

   Problem: Conflicting values
    \begin{itemize}
      \item different values derived from FITS
      \item various reasons (naming, incompatbility, bugs,...)
      \item could be avoided
    \end{itemize}

    \vspace{5pt}
    Goals:
    \begin{itemize}
      \item develop and apply post-processing methods to reduce conflicts
      \item add a logging mechanism (traceability)
    \end{itemize} 
\note{

}
\end{frame}

\section{Implementation}

\begin{frame}{Implementation}

  C3PO Workflow
      \begin{itemize}
        \item parsing the fits xml files with processing rules into elements
        \item pre-processing per metadata
        \item post-processing per element
      \end{itemize}
   
    \vspace{5pt}
    Drools
        \begin{itemize}
          \item rule framework
          \item powerful
          \item well documented
          \item widely used
        \end{itemize}
\note{

}
\end{frame}

\begin{frame}{Basic rule structure}

   \begin{itemize}
   \item \textbf{name}
   \begin{itemize}
	   \item identifies a rule
   \end{itemize}
   \item \textbf{salience}
   \begin{itemize}
   		\item defines the priority (or order) of rules
   		\item ranges from 0 to 1000 where 1000 is the maximum priority
   \end{itemize}
   \item \textbf{when} 
   \begin{itemize}
   		\item a list of (implicitly conjunctive) combined patterns
   		\item defines, when a rule is triggered
   \end{itemize}
   \item \textbf{then}
   \begin{itemize}
   		\item java-like imperative code 
   		\item defines, what to do, when the rule is triggered
   \end{itemize} 
   \end{itemize}
  
\note{

}
\end{frame}

\begin{frame}[fragile]
\frametitle{Example rule}

\lstset{xleftmargin=\parindent,basicstyle=\footnotesize\ttfamily,morecomment=[l]{//} }
\begin{lstlisting}

rule "set format GZIP to GZIP Format from Exiftool"
        salience 160
    when 
        $e : Element()
        $md : MetadataRecord(
            property.id == "format", 
            value == "GZIP",
            util.isFromTool(this, "Exiftool")
        ) from $e.metadata

    then
        String newValue = "GZIP Format";
        $md.setValue(newValue);
        modify ($e) {
            getId()
        }
end
\end{lstlisting}

\end{frame}

\begin{frame}{General Strategies}

   \begin{itemize}
   \item \textbf{update}
   \begin{itemize}
	   \item used when tools report ``unusual`` strings
	   \item value is simply changed
	   \item example: Jhove: ``in``, others: ``inches``
   \end{itemize}
   \item \textbf{ignore}
   \begin{itemize}
   		\item ignores a reported value, if it's known to be faulty
   		\item example: Exiftool reports ``XY``, while all others report "Portable Document Format``
   \end{itemize}
   \item \textbf{merge (and resolve)} 
   \begin{itemize}
   		\item used, when several tools agree on a value, but are marked as conflicting
   		\item the values are merged
   		\item if no differing records left,the conflict has been resolved
   \end{itemize}
   \end{itemize}
  
\note{

}
\end{frame}

\begin{frame}{Modification Logging}

   \begin{itemize}
   \item \textbf{Traceability}
   \begin{itemize}
	   \item which rules has been applied and in which order were they executed?
	   \item which values were changed or removed?
   \end{itemize}
   \item \textbf{Implementation}
   \begin{itemize}
   		\item RuleActivationListener: prints when a rule has been activated
   		\item ElementModificationListener: adds log entries to an element
   \end{itemize}
   \end{itemize}
  
\note{

}
\end{frame}

\section{Analysis}

\begin{frame}{Analysis - File Improvements}

\begin{table}[ht]
\begin{center}

\begin{tabular}[h]{l||r|r|r|r|r||r}
Files &  \multicolumn{5}{c}{Datasets} \\
        & 0 & 1 & 2 & 3 & 4 & Total \\
\hline
Total & 998 & 998 & 997 & 997 & 996 & 4986\\ 
\hline
with conflicts (pre) & 477 &	492 & 485 & 482	& 502 &	2438\\ 
with conflicts (post) & 249	& 243 &	240 & 253 &	259 & 1244\\
\hline
resolved & 228 & 249 & 245 & 229 & 243 & 1194

\end{tabular}
\end{center}
\label{tab:files}
\end{table}

  \note{

  }

\end{frame}

\begin{frame}{Analysis - Conflict Improvements}

Resolving over 53\% of conflicts.

\begin{table}[ht]
\begin{center}
\begin{tabular}[h]{l||r|r|r|r|r||r}
 &  \multicolumn{5}{c}{Datasets} \\
        & 0 & 1 & 2 & 3 & 4 & Total \\
\hline
conflicts & 1016 & 1010	& 1037 & 994 &1061 & 5118 \\
\hline
remaining conflicts & 483 &	447	& 472 &	479	& 496 &	2377

\end{tabular}
\end{center}
\label{tab:conflicts}
\end{table}

  \note{

  }

\end{frame}


\end{document}